\documentclass[Chinese,inNight]{CKBib}

\CKBIBsetup{
  title = {周四团契预查笔记}
}

\newcommand{\BibLength}{0.38\textwidth}
\newcommand{\TexLength}{0.58\textwidth}
\setcolumnwidth{\BibLength, \TexLength}
\setlength{\columnseprule}{0.5pt}

\newcommand{\queref}[1]{\hyperref[#1]{\ref*{#1}}}
\newcommand{\FormedWidth}{\paperwidth}
\newcommand{\FormedHeight}{\paperheight}
\newcommand{\Includepage}[2]{\includepdf[pages=#1,templatesize={\FormedWidth}{\FormedHeight},pagecommand={\thispagestyle{empty}\setcounter{page}{0}}]{#2}}

\begin{document}
\frontmatter
\pagenumbering{Roman}

\thispagestyle{empty}
\pagestyle{empty}
\Includepage{-}{cover.pdf}

\clearpage

\thispagestyle{empty}
\pagestyle{empty}

\begin{center}

\vspace*{0.2in}

\LARGE \SansF HCC UH小组

\vspace*{0.13in}

\HUGE \ExtBoldF{周四团契预查笔记}

\vspace*{0.18in}

\hrule

\vspace*{0.18in}

\Large 
\begin{minipage}{\textwidth}
\BibSentence{赛}{34}{16}{你们要查考宣读耶和华的书。这都无一缺少,无一没有伴偶,因为我的口已经吩咐,他的灵将他们聚集。}
\end{minipage}

\normalsize

\vspace*{0.68in}

\scshape UH Bible Study Group\\
Houston Chinese Church\\
\upshape \today
\end{center}

\clearpage

\begin{footnotesize}

\noindent 周四团契预查笔记,是辑录了晓士顿中国教会UH小组为周四福音团契预查工作的参考资料。\\
首次发行于2018年。

\vspace{1em}

\noindent 它遵守MIT License,任何人可以商用、私用、修改、分发,原作者不承担任何由此引发的权利、担保问题。

\end{footnotesize}

\begin{footnotesize}

\clearpage

\noindent MIT License

\vspace{1em}

\noindent Copyright (c) 2018 Notes for Fellowship on Thursday

\vspace{1em}

\noindent Permission is hereby granted, free of charge, to any person obtaining a copy of this software and associated documentation files (the "Software"), to deal in the Software without restriction, including without limitation the rights to use, copy, modify, merge, publish, distribute, sublicense, and/or sell copies of the Software, and to permit persons to whom the Software is furnished to do so, subject to the following conditions:

\noindent The above copyright notice and this permission notice shall be included in all copies or substantial portions of the Software.

\noindent THE SOFTWARE IS PROVIDED "AS IS", WITHOUT WARRANTY OF ANY KIND, EXPRESS OR IMPLIED, INCLUDING BUT NOT LIMITED TO THE WARRANTIES OF MERCHANTABILITY, FITNESS FOR A PARTICULAR PURPOSE AND NONINFRINGEMENT. IN NO EVENT SHALL THE AUTHORS OR COPYRIGHT HOLDERS BE LIABLE FOR ANY CLAIM, DAMAGES OR OTHER LIABILITY, WH-ETHER IN AN ACTION OF CONTRACT, TORT OR OTHERWISE, ARISING FROM, OUT OF OR IN CONNECTION WITH THE SOFTWARE OR THE USE OR OTHER DEALINGS IN THE SOFTWARE.

\end{footnotesize}

\clearpage
\hbox{}
\vspace*{\fill}
\thispagestyle{empty}
\newpage

\thispagestyle{empty}
\pagestyle{empty}

\renewcommand{\contentsname}{\chapFont 目录}
\tableofcontents*

\clearpage
%\hbox{}
%\vspace*{\fill}
%\thispagestyle{empty}
%\newpage

\thispagestyle{ruled}
\pagestyle{ruled}
\mainmatter

\chapter*{\chapFont 引子}
\setmark{引子}
\addcontentsline{toc}{chapter}{\chapFont 引子}

\BibSentence{太}{28}{18-20}{耶稣进前来,对他们说,天上,地下所有的权柄,都赐给我了。所以你们要去,使万民作我的门徒,奉父子圣灵的名,给他们施洗。(或作给他们施洗归于父子圣灵的名)凡我所吩咐你们的,都教训他们遵守,我就常与你们同在,直到世界的末了。}

感谢神,凭着祂奇妙的恩典、道路和真理,荫庇UH小组的不断成长。

本册志在提供一个有用的速查工具,以便UH小组的诸位同工不至遗漏在预查阶段所做的预备,从而为福音朋友传讲出更加准确、全面的真理。它可能不适合福音朋友直接阅读,如果条件允许,很希望它能被整理成真正的、面向福音性质的册子。

传讲福音需要我们做到真正的谦卑。神的真道何其浩然,凭我们一众凡人的理解,尚不能及其皮毛。但这真真切切的是我们的大使命,藉着神满有恩慈的应许,见证神救赎计划不断的成就。

在本章,我们希望能收录一些传讲福音的困惑和心得。在这项辑录工作发起之初,这里仍然是一片白纸,但是随着我们的共同成长,愿众人时时加添,将这里浇灌成参天大树。

\section*{编译要求}

本书采用{\LaTeXe}和{\XeTeX}编译,除此之外,还需要安装以下字体到\textbf{操作系统}中:

\SansF{思源黑体}\normalfont\selectfont: \href{https://blog.typekit.com/alternate/source-han-sans-chs/}{\Mundus};\LightF{思源宋体}\normalfont\selectfont: \href{https://source.typekit.com/source-han-serif/cn/}{\Mundus}

安装完成后,

\begin{enumerate}
  \item 执行\texttt{main.tex}编译一次。
  \item 执行项目目录下的\texttt{bibtex.bat},生成参考文献列表。
  \item 重新执行\texttt{main.tex},直到交叉引用警告全部消失即可。
\end{enumerate}

除此之外,还有一个版本,\texttt{singlechap.tex}是用来生成单章内容的。本工程使用的模板经过了特别设计,还能实现包括\textbf{夜间模式}的一些功能,如果需要,请查阅模板说明。

\section*{其他}

需要特别说明的是,后续的所有章节中,我们会使用如下不同的标记来区分、代表出不同时期添加的记录。例如,

\Tbib{这样的文字代表神的话。}

这样的文字代表预查时期添加的内容。

\Tafter{这样的文字代表在接下来的周四团契后添加的内容。}

\Tadd{这样的文字代表因为其他原因添加的内容。}

\cleardoublepage

\includechapter{docs/tex20180121}

\end{document}