\chapter{耶稣的论爱仇敌}
\setmark{耶稣论爱仇敌}

2018年2月4日 \hfill 路加福音第6章

\BibSentence{诗}{18}{25-27}{慈爱的人,你以慈爱待他;完全的人,你以完全待他;清洁的人,你以清洁待他;乖僻的人,你以弯曲待他。困苦的百姓,你必拯救;高傲的眼目,你必使他降卑。}

\section{查考的背景资料}

\begin{enumerate}
  \item 犹太人的血泪、仇恨和抗争史:
  \begin{enumerate}
    \item 正如\BibT[]{但}{2:39}的预言,犹太人“铜的腰和大腿”的时代结束了。\textit{Min}~\cite{book:minMessiah}指出,在“铁和泥的脚”(\BibT[]{但}{2:40})来到前后,犹太人经历的这段时间受到极大的逼迫,酝酿着不安定的社会情绪。
    \item \textbf{塞琉古王朝}(主前168年): 安条克四世(Antiochus IV Epiphanes)对犹太人进行了严重的逼迫。他洗劫圣所,屠杀平民,在圣殿山设立宙斯的偶像,禁止献祭和割礼,这样高压的统治给犹太人带来了巨大痛苦。\cite{wiki:Maccabean, wiki:Hasmonean}
    \item \textbf{马加比起义}(主前167-主前160年): 马加比兴起,反对塞琉古王朝,这场起义经历了七次主要战役。伴随着胜利,马加比激烈反对希腊化,破除偶像、恢复犹太传统,设立大祭司,同时也高压统治支持希腊化的犹太人\cite{wiki:Maccabean}。马加比起义带来的另一个重要传统是设立光明节\cite{wiki:Hanukkah},以纪念圣殿的洁净和光复。
    \item \textbf{哈斯莫尼王朝}(主前140-主前37年):马加比之后,犹太人有过一段短暂的独立。这段时间内既有扩张,也有内战,法利赛人和撒都该人也是在这段时间形成的。然而,这个王朝的统治者\textit{Hyrcanus}是希腊人,其与犹太人的隔阂和矛盾愈来愈难以弥合\cite{wiki:Hasmonean}。
    \item \textbf{希律王}(主前37年-主前4年):罗马人介入犹太国,并扶持的傀儡王。他兴修设施、恢复犹太国的繁荣,但依附罗马,严厉镇压反对者。在他在位期间,以色列民对弥赛亚的呼声达到了高峰\cite{wiki:Hasmonean}。
  \end{enumerate}
  \item 奋锐党:
  有激进、狂热主张的派系,敌视罗马,主张以暴力手段驱逐罗马人。在主后66-70年的犹太人起义(第一次犹太-罗马战争)声名大噪。使徒西门就被称谓“奋锐党人”,奋锐党的主张有极端化的倾向。\cite{wiki:Zealots}
\end{enumerate}

\section{正文预查}

\begin{paracol}{2}
  \litem{\BibT{路}{6}\BibS{20}{耶稣举目看着门徒,说:“你们贫穷的人有福了,因为{\Lord}的国是你们的!}}
  \ritem{“贫穷”(\emph{poor})在\BibT[]{太}{5:3}的对观是“虚心”(\emph{poor in spirit})。}
  \litem{\BibS{21}{你们饥饿的人有福了,因为你们将要饱足!你们哀哭的人有福了,因为你们将要喜笑!}}
  \ritem{“饥饿”(\emph{hunger})在\BibT[]{太}{5:6}的对观是“饥渴慕义”(\emph{hunger and thirst after righteousness})。和上一句一样,这里和《马太福音》的描述有物质和精神上不同侧重点的微妙差别,尽管可以将这里和《太》解释成完全相同的意思,但考虑到《路加福音》与前者的写作用意和出发点不同,也可以从侧重物质穷乏的方向来解释这里。}
  \litem{\BibS{22}{人为人子恨恶你们,拒绝你们,辱骂你们,弃掉你们的名,以为是恶,你们就有福了!}}
  \ritem{对观在\BibT[]{太}{5:11},这原因已经有启示告诉我们,\BibT{约}{15}\BibS{18}{世人若恨你们,你们知道(注:或作“该知道”)恨你们以先,已经恨我了。}\BibS{19}{你们若属世界,世界必爱属自己的;只因你们不属世界,乃是我从世界中拣选了你们,所以世界就恨你们。}所以我们要怎样做?《圣经》启示我们,\BibT{彼前}{3}\BibS{16}{存着无亏的良心,叫你们在何事上被毁谤,就在何事上可以叫那诬赖你们在基督里有好品行的人自觉羞愧。}}
  \litem{\BibS{23}{当那日,你们要欢喜跳跃,因为你们在天上的赏赐是大的!他们的祖宗待先知也是这样。}}
  \ritem{对观在\BibT[]{太}{5:12},摩西曾在旷野里被会众毁谤,引发了可拉一党的争执(\BibT[]{民}{16:1-40}),甚至就连亚伦都曾经毁谤过摩西(\BibT[]{民}{12:1})。类似地,但以理也曾受到严重的逼迫,被投入狮坑(\BibT[]{但}{6:16-18})。但这里主耶稣告诉我们,属世的逼迫和毁谤、拒绝并不可怕,祂也在告诉我们如何面对我们的失丧。尽管我们贫穷、饥渴、受到逼迫,祂却给我们极大的安慰,应许神会凭着公义将平安和喜乐赐给我们。}
  \litem{\BibS{24}{但你们富足的人有祸了,因为你们受过你们的安慰!}}
  \ritem{}
  \litem{\BibS{25}{你们饱足的人有祸了,因为你们将要饥饿!你们喜笑的人有祸了,因为你们将要哀恸哭泣!}}
  \ritem{这里并非是字面意思上反对人富足、饱足和喜笑,神也不是在这里给我们定道德标准,要我们过苦行的生活。而是在警醒人要有谦卑的姿态,正如\BibT{雅}{4}\BibS{9}{你们要愁苦、悲哀、哭泣,将喜笑变作悲哀,欢乐变作愁闷。}\BibS{9}{务要在主面前自卑,主就必叫你们升高。},不可忘了自己在神面前的渺小,而肆意妄为,\BibT{启}{17}\BibS{2}{地上的君王与她行淫,住在地上的人喝醉了她淫乱的酒。”}最后难免喝“神大怒的酒”。}
  \litem{\BibS{26}{人都说你们好的时候,你们就有祸了,因为他们的祖宗待假先知也是这样!}}
  \ritem{人肉体是软弱的,当有假先知扭曲神的道时,因着这异端随从罪性,人就受害了。\BibT{彼后}{2}\BibS{1}{从前在百姓中有假先知起来,将来在你们中间也必有假师傅,私自引进陷害人的异端,连买他们的主他们也不承认,自取速速的灭亡。}不认识神的人,更容易受欺骗,\BibT{约壹}{4}\BibS{5}{他们是属世界的,所以论世界的事,世人也听从他们。}然则,神早已启示我们两条门路,\BibT{太}{7}\BibS{13}{你们要进窄门。因为引到灭亡,那门是宽的,路是大的,进去的人也多;}\BibS{14}{引到永生,那门是窄的,路是小的,找着的人也少。}不要因着路宽、好走就随从世界走上那大路。}
  \litem{\BibS{27}{只是我告诉你们这听道的人,你们的仇敌,要爱他!恨你们的,要待他好!}}
  \ritem{谁是我们的仇敌?从后文来看,这不是深仇大恨的仇敌,而是在任何事、哪怕是小事上得罪我们、恨我们的仇敌。谁是神的仇敌?至少我们知道,\BibT{罗}{8}\BibS{7}{原来体贴肉体的,就是与{\Lord}为仇,因为不服{\Lord}的律法,也是不能服。}
  
  犹太社区在这最黑暗的时候,弥漫着苦毒的氛围,报仇的呼声暗流汹涌,但主告诉听道的人,弥赛亚的道不是这样的,不是要以恶报恶,能胜恶的不是恶,而是善。正如启示\BibT{罗}{12}\BibS{17}{不要以恶报恶。众人以为美的事,要留心去做。}\BibS{18}{若是能行,总要尽力与众人和睦。}\BibS{19}{亲爱的弟兄,不要自己伸冤,宁可让步,听凭主怒(注:或作“让人发怒”)。因为经上记着:“主说:‘伸冤在我,我必报应。’”}\BibS{20}{所以,“你的仇敌若饿了,就给他吃;若渴了,就给他喝。因为你这样行,就是把炭火堆在他的头上。”}\BibS{21}{你不可为恶所胜,反要以善胜恶。}

  大卫对待扫罗,也是这样的道理,扫罗与他为仇,他却因扫罗是神的受膏者,在神将扫罗交给他的时候不肯害扫罗,反而屡次救他性命(\BibT[]{撒上}{24, 26})。扫罗已经在神的面前有亏,他是仇敌、也谈不上是义人,但是大卫把主权交给神,而不是自己去和扫罗争斗,因为他很清楚神要讨取扫罗的恶(\BibT[]{撒上}{26:10}),所以他爱惜扫罗不是以触犯神的公义为前提的。这种观念总结而言就是\BibT{诗}{37}\BibS{20}{恶人却要灭亡。耶和华的仇敌要像羊羔的脂油(注:或作“像草地的华美”),他们要消灭,要如烟消灭。}\BibS{21}{恶人借贷而不偿还;义人却恩待人,并且施舍。}}
  \litem{\BibS{28}{咒诅你们的,要为他祝福!凌辱你们的,要为他祷告!}}
  \ritem{这是主耶稣的作为:\BibT{路}{23}\BibS{34}{当下耶稣说:“父啊,赦免他们!因为他们所做的,他们不晓得。”……}}
  \litem{\BibS{29}{有人打你这边的脸,连那边的脸也由他打。有人夺你的外衣,连里衣也由他拿去。}}
  \ritem{这是主耶稣的作为:\BibT{约}{19}\BibS{23}{兵丁既然将耶稣钉在十字架上,就拿他的衣服分为四份,每兵一份;又拿他的里衣,这件里衣原来没有缝儿,是上下一片织成的。}\BibS{24}{他们就彼此说:“我们不要撕开,只要拈阄,看谁得着。”这要应验经上的话说:“他们分了我的外衣,为我的里衣拈阄。”兵丁果然做了这事。}}
  \litem{\BibS{30}{凡求你的,就给他。有人夺你的东西去,不用再要回来。}}
  \ritem{}
  \litem{\BibS{31}{你们愿意人怎样待你们,你们也要怎样待人。}}
  \ritem{}
  \litem{\BibS{32}{你们若单爱那爱你们的人,有甚么可酬谢的呢?就是罪人也爱那爱他们的人。}}
  \ritem{}
  \litem{\BibS{33}{你们若善待那善待你们的人,有甚么可酬谢的呢?就是罪人也是这样行。}}
  \ritem{}
  \litem{\BibS{34}{你们若借给人,指望从他收回,有甚么可酬谢的呢?就是罪人也借给罪人,要如数收回。}}
  \ritem{}
  \litem{\BibS{35}{你们倒要爱仇敌,也要善待他们,并要借给人不指望偿还,你们的赏赐就必大了,你们也必作至高者的儿子,因为他恩待那忘恩的和作恶的。}}
  \ritem{主已经用祂的作为告诉我们,身为神的儿子,祂甚至把恩慈赐给那些得罪祂的人。可见,以上的教导,主耶稣讲论到他自己了。但祂同时也期待我们行出祂的生命,学习祂的样式,所以将“神的儿女”应有的姿态展现给我们,启示给我们,明言给我们。这段教导归于“谅解”上。
  
  但需要注意的是,我们应当原谅别人,却不能期待别人原谅我们,而要主动争取别人的原谅。\BibT{太}{5}\BibS{23}{所以,你在祭坛上献礼物的时候,若想起弟兄向你怀怨,}
  \BibS{24}{就把礼物留在坛前,先去同弟兄和好,然后来献礼物。}\BibS{25}{你同告你的对头还在路上,就赶紧与他和息,恐怕他把你送给审判官,审判官交付衙役,你就下在监里了。}这说来容易做来难,但仍可以成为我们仰望的目标,因为人的度量是锻炼出来的,藉着属灵的成长,做到不怀恨,不再来记恨,行出谅解人的教训来。}
  \litem{\BibS{36}{你们要慈悲,像你们的父慈悲一样。}}
  \ritem{我们本身都是罪人,从前与神为仇,神却仍然愿意原谅我们,主动与我们和好,\BibT{罗}{5}\BibS{9}{现在我们既靠着他的血称义,就更要借着他免去{\Lord}的忿怒。}\BibS{10}{因为我们作仇敌的时候,且借着{\Lord}儿子的死得与{\Lord}和好;既已和好,就更要因他的生得救了。}\BibS{11}{不但如此,我们既借着我主耶稣基督得与{\Lord}和好,也就借着他以{\Lord}为乐。}
  
  故而,我们应当记念着神给我们的救恩,并且对还没有得救的人怀着帮助的心,使他们早早归向基督。}
  \litem{\BibS{37}{你们不要论断人,就不被论断;你们不要定人的罪,就不被定罪;你们要饶恕人,就必蒙饶恕(注:“饶恕”原文作“释放”);}}
  \ritem{我们不能论断人,乃是因为我们没有能力论断。律法使我们知罪,因此谁也没法自陈无罪,做那个判行淫的妇女死刑的人(\BibT[]{约}{8:1-9}),责人当责己,如\BibT[]{罗}{2:17-24},,我们怎样责备别人,罪就一样控告着我们自己,因此须要谦卑,把审判的主权归给神,以免僭越主权,亵渎神的名。}
  \litem{\BibS{38}{你们要给人,就必有给你们的,并且用十足的升斗,连摇带按,上尖下流地倒在你们怀里;因为你们用甚么量器量给人,也必用甚么量器量给你们。”}}
  \ritem{}
\end{paracol}

\qquad

\renewcommand{\bibname}{本章参考}
\bibliographystyle{IEEEtran}
\bibliography{bib/tex201802}