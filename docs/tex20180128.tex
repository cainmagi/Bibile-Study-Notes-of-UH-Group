\chapter{耶稣受洗、被试探}
\setmark{耶稣受洗、被试探}

2018年1月21日 \hfill 路加福音第3、4章

\BibSentence{可}{10}{39}{他们说:“我们能。”耶稣说:“我所喝的杯,你们也要喝;我所受的洗,你们也要受;}

\BibSentence{来}{4}{15}{因我们的大祭司,并非不能体恤我们的软弱。他也曾凡事受过试探,与我们一样。只是他没有犯罪。}

\section{查考的背景资料}

\begin{enumerate}
  \item \textbf{关于三一神论}: 三一神论(Trinity)是一个复杂的问题。实际上,如果为了得到一个信实、可靠的结论,可以直接参考尼西亚信经\cite{book:NiceneLSB}和迦克墩信经\cite{book:ChalcedonBCP}。本卷附将基于\cite{book:sproul2011trinity}专门查考这一学说,为了确保能传讲正统、无误的三一神论,防止陷入异端,可以简要地概括成以下几个要点:
  \begin{enumerate}
    \item \textbf{奥秘性}: 神的智慧高于人的智慧的明确彰显之一。人只能从《圣经》的启示里查考到三一神论的确据,却无法推导它存在的理由,试图揣测神为何以三一神的形式示人的做法,都难免陷入以自我为中心解释神意的局限和谬误之中。
    \item \textbf{同质性}: 即“三”的意义。圣子由圣父所生,圣灵源自圣父和圣子\footnote{关于这一点,东方教会认为圣灵只来自于圣父。}。但是,这里“受生”绝非受造,三个位格是同质同等、同尊同荣的,祂们都是自有永有的,并不因这种诞生的关系使得在某时刻,某个位格(person)尚未存在。
    \item \textbf{独一性}: 即“一”的意义。三个位格同属于独一神,祂们彼此之间不能分离,不是三位神(Tritheism);三个位格也不等同于同一位格,换言之,例如,如果认为圣父和圣子是同一位格在不同时的不同表现形式,那么论及“道成肉身”时就不得不将“道”和神分离,反而破坏了独一性。
    \item \textbf{基督学}: 与其他两个位格不同,基督兼具有真正的神性(nature)和真正的人性。祂既不是两个位格的结合(有一个“神格”和另一个“人格”),也不是“半神半人”的混成。从神性方面看祂,祂是真正的神,从人性方面看祂,祂是真正的人。
  \end{enumerate}
  \item \textbf{关于洗礼}: 浸礼/洗礼(baptism)是主耶稣亲自命定的圣事,如何看待洗礼,大致有以下的总结,本卷同样也会在附中基于\cite{book:sproul2011baptism}专门查考这一圣事所代表的意义。就总结而言,可以认为洗礼有如下特征:
  \begin{enumerate}
    \item \textbf{主的洗不同于约翰的洗}: 施洗约翰所传的是悔改的洗,主所赐的洗不单单要人悔改,还代表着个人信主的宣告和见证。
    \item \textbf{洗礼不是救恩所必须的}: 人得到救恩纯出于信心,使得我们以有罪之身仍能领受神的大恩。洗礼不能为我们提供任何除罪的功效。
    \item \textbf{洗礼是基督徒必须受的}: 洗礼出于主命,藉着洗礼宣告加入新约设立的教会,主耶稣的身体这一大家庭(神的家庭)。因此,对于基督徒,抗拒受洗即抗拒主命,是不合宜的。
    \item \textbf{洗礼是神赐重生的见证}: 虽然洗礼不带来重生,但它确实见证了信徒的重生,宣告了我们与基督同死同复活。作为新约命定的仪式,它代表了神要藉着圣灵洁净罪人的伟大承诺。因此,受洗虽然不是救恩必须的,却见证了神的信实,和我们心怀喜悦地接纳神的救恩。
  \end{enumerate}
\end{enumerate}

\section{正文预查}

\begin{paracol}{2}
  \litem{\BibT{路}{3}\BibS{21}{众百姓都受了洗,耶稣也受了洗。正祷告的时候,天就开了,}}
  \ritem{约翰的洗是悔改的洗,前文提到\BibT{路}{3}\BibS{2}{……约翰在旷野里,{\Lord}的话临到他。}\BibS{3}{他就来到约旦河一带地方,宣讲悔改的洗礼,使罪得赦。}而,另一方面,又有\BibT{来}{4}\BibS{15}{……他也曾凡事受过试探,与我们一样;只是他没有犯罪。}因此,耶稣是以无罪之身受悔改的洗的。
  
  这一点就连约翰也不能理解,\BibT{太}{3}\BibS{18}{约翰想要拦住他,说:“我当受你的洗,你反倒上我这里来吗?”}但是,\BibT{太}{3}\BibS{19}{耶稣回答说:“你暂且许我,因为我们理当这样尽诸般的义(注:或作“礼”)。 ”于是约翰许了他。}

  主耶稣的回答表明,祂所受洗乃是为成全神的义。正如同祂以后以无罪之身钉十字架一样,祂不须悔改,却为我们的悔改而受洗。同时,祂也提到“我们理当尽诸般的义”,并亲自作出表率,在祂还未出来传道的时候,就已经开始启示真理和恩典给我们了。}
  \litem{\BibS{22}{圣灵降临在他身上,形状仿佛鸽子;又有声音从天上来,说:“你是我的爱子,我喜悦你。”}}
  \ritem{这是一段令人感到危险的经文,因为这里出现圣灵“形状仿佛鸽子”的比喻(\cite{book:smith2009not}讨论了圣灵的各种象征)。如何不损害神的荣耀来理解这个问题,在下面的阐述里可能还嫌不足:
  
  1. 这里描述“圣灵的形状”,最重要的是在强调圣灵是可见的,神的声音也是可闻的。约翰、并周围受洗的人,都为主耶稣被圣灵充满作了见证,这经文也为主被圣灵充满作了见证。它不是一种看不见、摸不着的过程,而是明确无疑的启示。
  
  2. 这里“形状仿佛鸽子”,并非在论及圣灵的形象是什么,不是说,圣灵的样子就像一只鸽子,或者有鸽子的样子。查考希腊文,比较好的翻译是``in bodily form'',这是两个词,``bodily''和``form''都是形容词而非副词,因此它是一种诠释,一种片面的描述,而不是对神的形象下断语。一个较好的参照是\BibT{创}{2}\BibS{18}{神的灵运行在水面上。},故而此处不是在说,神(圣灵)的形象像鸽子,而是在描述,圣灵飞行、降落的姿态像鸽子一样。类似地,\textit{Stein}~\cite{book:stein1993luke}也不支持圣灵形象是鸽子(take form);类似前述的解释同样是\textit{Bock}~\cite{book:bock1994luke}所列举的几种解释中、不那么生硬的一种,但\textit{Bock}认为它仍然略显牵强。
  
  3. 一个容易陷入谬误的见解是,从这段经文将鸽子和神联系起来。例如\cite{book:smith2009not, blog:Luke3BibltStudy, book:vision2017Catho}都主张鸽子是神的表记、或象征,并进而引用《圣经》,试图发现鸽子像神的美好、圣洁的特征来。然而神用造物将自己(的行为)启示给人,是为了让人能理解。但我们始终要铭记的是,无论神用什么样的造物来比喻、启示,只要神没有明确启示这造物分别为圣、是圣洁的,我们就不能擅自推断出这样的结论,更不要说将这造物和神本身联系起来。例如,\BibT{出}{19}\BibS{4}{‘我向埃及人所行的事,你们都看见了,且看见我如鹰将你们背在翅膀上,带来归我。}但是,神已经在律法启示,不洁净的动物有:\BibT{利}{11}\BibS{16}{鸵鸟、夜鹰、鱼鹰、鹰与其类;}。可见,前面的“如鹰”不过是一个单纯的比喻而已,超乎这个比喻之上什么意义也没有。我们对待神启示自身的经节也要持以谨慎和敬畏的态度,以免如同\BibT{利}{20}\BibS{23}{你们不可做甚么神像与我相配,不可为自己做金银的神像。},在心中用造物的形象来配神的形象。}
  \litem{\BibS{23}{耶稣开头传道,年纪约有三十岁。依人看来,他是约瑟的儿子,约瑟是希里的儿子,}}
  \ritem{“依人看来”,有双重意思。它既表示约瑟并非耶稣的生父,也表示希里并非约瑟的生父。实际上,希里是马利亚的生父,约瑟因为和马利亚婚配,从而成为“希里的儿子”。由此结合\BibT[]{太}{1:1-17},确定了耶稣由(养)父辈、母辈追溯到“亚伯拉罕的后裔,大卫的子孙”和“女人的后裔”的正统性。}
  \litem{\BibT{路}{4}\BibS{1}{耶稣被圣灵充满,从约旦河回来,圣灵将他引到旷野,四十天受魔鬼的试探。}}
  \ritem{这里有两个值得讨论的问题。其一是圣灵充满,其二是试探的意义。
    
  在旧约就已经有不少被圣灵充满的例证。例如比撒列\BibT{出}{35}\BibS{31}{又以{\Lord}的灵充满了他,使他有智能、聪明、知识,能做各样的工;},但以理\BibT{但}{4}\BibS{8}{末后,那照我神的名,称为伯提沙撒的但以理来到我面前,他里头有圣神的灵……}。与新约不同的是,这里圣灵充满给予人智慧和能力,完成神交托的重要的工作,不会长久驻留在人身上。例如扫罗不守诫命,神厌弃了他,于是\BibT{撒上}{16}\BibS{14}{耶和华的灵离开扫罗,有恶魔从耶和华那里来扰乱他。}
    
  试探和试炼的区别,见于唐崇荣牧师\cite{blog:TrailTang}的解释。\BibT{雅}{1}\BibS{13}{人被试探,不可说:“我是被{\Lord}试探”,因为{\Lord}不能被恶试探,他也不试探人。}试探虽然出于人的罪和魔鬼,但是神能保护被试探的人,把恶意的试探转变成善意的试炼,把魔鬼毁坏人的目的转为造就人的试炼。
  
  全段讨论耶稣受试探,可以从两个角度入手来看待这一段经文。
  
  1. 与亚当和夏娃所受的试探类比,可以看出在此耶稣和人的始祖所面临的处境是完全同等的,彰显出耶稣的人性,但主耶稣的义坚不可摧。始祖没能抵挡住试探,暴露出自己悖逆神的恶性,但主耶稣自始至终都顺服神的教导,祂展示给我们一个真正的人的形象,更展示给我们什么是真正的义人,从而作出了表率。
  
  2. 从律法的角度来看,耶稣受试探显示他顺服律法的智慧。以下魔鬼的每一个试探都在诱使主耶稣违背神的诫命,但主既成为人的样式,就完完全全顺服在神的律法下,在魔鬼的面前得胜。故而,他知道我们行出律法的难处,体察我们,我们靠自己不能行出义,但依靠主就能改变自己的生命,顺服神的教导。
  }
  \litem{\BibS{2}{那些日子没有吃甚么,日子满了,他就饿了。}}
  \ritem{}
  \litem{\BibS{3}{魔鬼对他说:“你若是{\Lord}的儿子,可以吩咐这块石头变成食物。”}}
  \ritem{}
  \litem{\BibS{4}{耶稣回答说:“经上记着说:‘人活着不是单靠食物,乃是靠{\Lord}口里所出的一切话。’”}}
  \ritem{律法有载:\BibT{申}{8}\BibS{3}{他苦炼你,任你饥饿,将你和你列祖所不认识的吗哪赐给你吃,使你知道人活着不是单靠食物,乃是靠耶和华口里所出的一切话。}
  
  神曾经领以色列民出埃及,但是全会众一遇到饥饿、口渴的境地就忍不住向神抱怨,第一次(\BibT[]{出}{15:24}),神把苦水变甜;第二次(\BibT[]{出}{16:2-3}),神赐下吗哪;第三次(\BibT[]{出}{17:3}),神让磐石里流出水来。此后神赐下法柜和法版,会众都承认了神的典章,但这抱怨还没有停止。第四次(\BibT[]{民}{11:4-6}),百姓嫌弃神的食物寡淡,神赐下食物,但也为他们贪欲的心施下了严厉的惩罚;第五次(\BibT[]{民}{20:3-5}),神替会众解渴,却也褫夺了摩西、亚伦这一代人进入应许之地的机会。这中间因为其他原因,会众试探神的次数不胜枚举。

  因此,看起来这里好像只是在怂恿主耶稣运用神的权能变出食物,以证明自己的神性,但如果主耶稣从了魔鬼的计谋,就无异于否定神的公义,以及在过往对以色列民为了食物抱怨的判决。主在这里的回答是精准、合宜的。以色列民为食物抱怨了五次,但主一次都没有过。}
  \litem{\BibS{5}{魔鬼又领他上了高山,霎时间把天下的万国都指给他看,}}
  \ritem{}
  \litem{\BibS{6}{对他说:“这一切权柄、荣华,我都要给你,因为这原是交付我的,我愿意给谁就给谁。}}
  \ritem{这里魔鬼指认天下的权柄在自己手上,是真话还是谎话?
  
  魔鬼虽然爱说谎,但从某种程度上,它这里讲得并没有错。因为从亚当夏娃从它的计谋犯罪开始,全天下就都被罪恶捆绑了,\BibT{约壹}{5}\BibS{19}{我们知道我们是属神的,全世界都卧在那恶者手下。}但这种权柄是属于阴间的,并不长久,因为归根结底权柄是神的,\BibT{诗}{22}\BibS{18}{因为国权是耶和华的;他是管理万国的。}魔鬼自以为大,为辖制住人得意,但到审判的时候,\BibT{启}{20}\BibS{14}{死亡和阴间也被扔在火湖里,这火湖就是第二次的死。}
  }
  \litem{\BibS{7}{你若在我面前下拜,这都要归你。”}}
  \ritem{}
  \litem{\BibS{8}{耶稣说:“经上记着说:当拜主你的{\Lord},单要事奉他。”}}
  \ritem{律法有载:\BibT{申}{10}\BibS{20}{你要敬畏耶和华你的 神,事奉他,专靠他,也要指着他的名起誓。}
  
  这就是尊重神的主权,它合乎十诫的第一条(\BibT[]{出}{20:2-3})。主耶稣也明确给出教导,最大的诫命是\BibT{太}{22}\BibS{37}{耶稣对他说:“你要尽心、尽性、尽意,爱主你的 神。}不为权势所诱惑,也不屈服与权势,单单顺服神和祂的旨意,主耶稣为我们作出了表率。}
  \litem{\BibS{9}{魔鬼又领他到耶路撒冷去,叫他站在殿顶(注:“顶”原文作“翅”)上,对他说:“你若是{\Lord}的儿子,可以从这里跳下去;}}
  \ritem{}
  \litem{\BibS{10}{因为经上记着说:{}‘主要为你吩咐他的使者保护你。}}
  \ritem{}
  \litem{\BibS{11}{他们要用手托着你,免得你的脚碰在石头上。’”}}
  \ritem{这里显示魔鬼熟悉《圣经》,却同时狡猾地删改了原文,\BibT{诗}{91}\BibS{11}{因他要为你吩咐他的使者,在你行的一切道路上保护你。}\BibS{12}{他们要用手托着你,免得你的脚碰在石头上。}它故意去掉了“在你行的一切道路上”,显然这和主站在殿顶的情况是不符的,魔鬼却用损害《圣经》原意的形式,来试探主。}
  \litem{\BibS{12}{耶稣对他说:“经上说:‘不可试探主你的{\Lord}。’”}}
  \ritem{但同时须知的是,即使不考虑与原意的出入,主的看护是时时同在的,也不能从魔鬼的计谋,因为主已经给出答案。律法有载:\BibT{申}{6}\BibS{16}{你们不可试探耶和华你们的{\Lord},……}
  
  相对地,以色列民却\BibT{民}{14}\BibS{22}{这些人虽看见我的荣耀和我在埃及与旷野所行的神迹,仍然试探我这十次,不听从我的话,}在新约里,我们还能陆续见到他们不断试探主,这从不信和罪中来的试探,实在是过于致命。实际上,任何时候,我们思想“神现在没有与我同在”或者“神掩面没有看我”,都是在试探神。而这种自我欺骗的后果,往往就是心里不信、行为上也干犯神。常常思想神的无所不在,常常怀有对神的敬畏,不站罪人的道路,主耶稣已经为我们作出表率。}
  \litem{\BibS{13}{魔鬼用完了各样的试探,就暂时离开耶稣。}}
  \ritem{我们重新思想这段经节:\BibT{来}{4}\BibS{15}{……他也曾凡事受过试探,与我们一样;只是他没有犯罪。}
  
  试探临到时,主用律法的教导战胜了魔鬼,这是我们这些被罪辖制的人所做不到的。但它显示出律法真的是公义、正直、可喜爱的,令我们不由赞美主有福:\BibT{诗}{1}\BibS{2}{惟喜爱耶和华的律法,昼夜思想,这人便为有福!}
  
  主抵挡住了试探,不但没有犯罪,还教我们祷告,\BibT{路}{11}\BibS{4}{……不叫我们遇见试探;救我们脱离凶恶(注:有古卷无末句)}。魔鬼狡猾、凶恶,我们在它面前真的软弱无力。可是,主不但战胜了它,还愿做我们的盾牌,保我们“不受试探”。当试探真的临到的时候,真应当向主这样祷告,藉着对主的信心来战胜它,甚至像约伯一样,将试探变为造就的试炼,将属世的咒诅变为属灵的祝福。}
\end{paracol}

\qquad

\renewcommand{\bibname}{本章参考}
\bibliographystyle{IEEEtran}
\bibliography{bib/tex201801}