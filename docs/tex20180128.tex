\chapter{耶稣受洗、被试探}
\setmark{耶稣受洗、被试探}

2018年1月28日 \hfill 路加福音第3、4章

\BibSentence{可}{10}{39}{他们说:“我们能。”耶稣说:“我所喝的杯,你们也要喝;我所受的洗,你们也要受;}

\BibSentence{来}{4}{15}{因我们的大祭司,并非不能体恤我们的软弱。他也曾凡事受过试探,与我们一样。只是他没有犯罪。}

\section{查考的背景资料}

\begin{enumerate}
  \item \textbf{关于三一神论}: 三一神论(Trinity)是一个复杂的问题。实际上,如果为了得到一个信实、可靠的结论,可以直接参考尼西亚信经\cite{book:NiceneLSB}和迦克墩信经\cite{book:ChalcedonBCP}。本卷附将基于\cite{book:sproul2011trinity}专门查考这一学说,为了确保能传讲正统、无误的三一神论,防止陷入异端,可以简要地概括成以下几个要点:
  \begin{enumerate}
    \item \textbf{奥秘性}: 神的智慧高于人的智慧的明确彰显之一。人只能从《圣经》的启示里查考到三一神论的确据,却无法推导它存在的理由,试图揣测神为何以三一神的形式示人的做法,都难免陷入以自我为中心解释神意的局限和谬误之中。
    \item \textbf{同质性}: 即“三”的意义。圣子由圣父所生,圣灵源自圣父和圣子\footnote{关于这一点,东方教会认为圣灵只来自于圣父。}。但是,这里“受生”绝非受造,三个位格是同质同等、同尊同荣的,祂们都是自有永有的,并不因这种诞生的关系使得在某时刻,某个位格(person)尚未存在。
    \item \textbf{独一性}: 即“一”的意义。三个位格同属于独一神,祂们彼此之间不能分离,不是三位神(Tritheism);三个位格也不等同于同一位格,换言之,例如,如果认为圣父和圣子是同一位格在不同时的不同表现形式,那么论及“道成肉身”时就不得不将“道”和神分离,反而破坏了独一性。
    \item \textbf{基督学}: 与其他两个位格不同,基督兼具有真正的神性(nature)和真正的人性。祂既不是两个位格的结合(有一个“神格”和另一个“人格”),也不是“半神半人”的混成。从神性方面看祂,祂是真正的神,从人性方面看祂,祂是真正的人。
  \end{enumerate}
  \item \textbf{关于洗礼}: 浸礼/洗礼(baptism)是主耶稣亲自命定的圣事,如何看待洗礼,大致有以下的总结,本卷同样也会在附中基于\cite{book:sproul2011baptism}专门查考这一圣事所代表的意义。就总结而言,可以认为洗礼有如下特征:
  \begin{enumerate}
    \item \textbf{主的洗不同于约翰的洗}: 施洗约翰所传的是悔改的洗,主所赐的洗不单单要人悔改,还代表着个人信主的宣告和见证。
    \item \textbf{洗礼不是救恩所必须的}: 人得到救恩纯出于信心,使得我们以有罪之身仍能领受神的大恩。洗礼不能为我们提供任何除罪的功效。
    \item \textbf{洗礼是基督徒必须受的}: 洗礼出于主命,藉着洗礼宣告加入新约设立的教会,主耶稣的身体这一大家庭(神的家庭)。因此,对于基督徒,抗拒受洗即抗拒主命,是不合宜的。
    \item \textbf{洗礼是神赐重生的见证}: 虽然洗礼不带来重生,但它确实见证了信徒的重生,宣告了我们与基督同死同复活。作为新约命定的仪式,它代表了神要藉着圣灵洁净罪人的伟大承诺。因此,受洗虽然不是救恩必须的,却见证了神的信实,和我们心怀喜悦地接纳神的救恩。
  \end{enumerate}
\end{enumerate}

\section{学习目标}

\begin{enumerate}
  \item 知道耶稣凡事受过试探,只是没有犯罪;
  \item 明白神话语的重要性,进来渴慕神的话语;
  \item 意识到那位凡事受过试探没有犯罪的耶稣,为我们的罪死在十字架上;
\end{enumerate}

\section{大纲}

\begin{itemize}
  \item \BibT{v.}{1-2}\smallTitle{主耶稣在旷野受试探}:这里讲到主耶稣被圣灵充满,并且圣灵将祂引到旷野,在那里受到魔鬼的试探;
  \begin{itemize}
    \item 这里提到的40天可以对照以色列在旷野漂泊的40年(\BibT[]{申}{8:2})以及摩西在西奈山40昼夜不吃不喝。(\BibT[]{出}{24:18});
  \end{itemize}
  \item \BibT{v.}{3-4}\smallTitle{魔鬼试探主耶稣让他将石头变成食物}:主回应「经上记著说:『人活著不是单靠食物,乃是靠神口里所出的一切话。』」;
  \begin{itemize}
    \item 主耶稣的回应来自申命记\BibT[]{申}{8:3};
    \item 应用问题:“人活著不是单靠食物,乃是靠神口里所出的一切话。” 这句话告诉我们什么。或者说我们可以学习到什么? 
    \item 读以下经文:
    \begin{itemize}
      \item \BibT{诗}{119}\BibS{11}{我将你的话藏在心里,免得我得罪你。}
      \item \BibT{诗}{119}\BibS{97}{我何等爱慕你的律法,终日不住地思想。}
      \item \BibT{诗}{119}\BibS{98}{你的命令常存在我心里,使我比仇敌有智慧。}
      \item \BibT{诗}{119}\BibS{105}{你的话是我脚前的灯,是我路上的光。}
    \end{itemize}
  \end{itemize}
  \item \BibT{v.}{5-8}\smallTitle{魔鬼试探主耶稣,让主向牠下拜}:主耶稣的回应:「经上记著说:当拜主你的神,单要事奉他。」;
  \begin{itemize}
    \item 主耶稣的回应来自\BibT[]{申}{6:13};
    \item 举一个例子(王明道的例子)来强调经上记著说:当拜主你的神,单要事奉他。
    \item 应用问题:当我们面对实际问题时,我们是否为了五斗米折腰?
  \end{itemize}
  \item \BibT{v.}{9-12}\smallTitle{魔鬼试探主让祂从殿顶跳下去}:主耶稣的回应:「经上说:『不可试探主你的神。』」;
  \begin{itemize}
    \item 魔鬼的话引自\BibT[]{诗}{91:11-12};
    \item 主耶稣的回应来自\BibT[]{申}{6:16};
    \item 魔鬼了解圣经,时常也通过错误地使用圣经来试探我们,并且\BibT{彼前}{5}\BibS{8}{务要谨守,警醒。因为你们的仇敌魔鬼,如同吼叫的狮子,遍地游行,寻找可吞吃的人。}我们应该怎么回应?
  \end{itemize}
\end{itemize}

\section{正文预查}

\begin{paracol}{2}
  \litem{\BibT{路}{3}\BibS{21}{众百姓都受了洗,耶稣也受了洗。正祷告的时候,天就开了,}}
  \ritem{约翰的洗是悔改的洗,前文提到\BibT{路}{3}\BibS{2}{……约翰在旷野里,{\Lord}的话临到他。}\BibS{3}{他就来到约旦河一带地方,宣讲悔改的洗礼,使罪得赦。}而,另一方面,又有\BibT{来}{4}\BibS{15}{……他也曾凡事受过试探,与我们一样;只是他没有犯罪。}因此,耶稣是以无罪之身受悔改的洗的。
  
  这一点就连约翰也不能理解,\BibT{太}{3}\BibS{18}{约翰想要拦住他,说:“我当受你的洗,你反倒上我这里来吗?”}但是,\BibT{太}{3}\BibS{19}{耶稣回答说:“你暂且许我,因为我们理当这样尽诸般的义(注:或作“礼”)。 ”于是约翰许了他。}

  主耶稣的回答表明,祂所受洗乃是为成全神的义。正如同祂以后以无罪之身钉十字架一样,祂不须悔改,却为我们的悔改而受洗。同时,祂也提到“我们理当尽诸般的义”,并亲自作出表率,在祂还未出来传道的时候,就已经开始启示真理和恩典给我们了。}
  \litem{\BibS{22}{圣灵降临在他身上,形状仿佛鸽子;又有声音从天上来,说:“你是我的爱子,我喜悦你。”}}
  \ritem{这是一段令人感到危险的经文,因为这里出现圣灵“形状仿佛鸽子”的比喻(\cite{book:smith2009not}讨论了圣灵的各种象征)。如何不损害神的荣耀来理解这个问题,在下面的阐述里可能还嫌不足:
  
  1. 这里描述“圣灵的形状”,最重要的是在强调圣灵是可见的,神的声音也是可闻的。约翰、并周围受洗的人,都为主耶稣被圣灵充满作了见证,这经文也为主被圣灵充满作了见证。它不是一种看不见、摸不着的过程,而是明确无疑的启示。
  
  2. 这里“形状仿佛鸽子”,并非在论及圣灵的形象是什么,不是说,圣灵的样子就像一只鸽子,或者有鸽子的样子。查考希腊文,比较好的翻译是``in bodily form'',这是两个词,``bodily''和``form''都是形容词而非副词,因此它是一种诠释,一种片面的描述,而不是对神的形象下断语。一个较好的参照是\BibT{创}{2}\BibS{18}{{\Lord}的灵运行在水面上。},故而此处不是在说,神(圣灵)的形象像鸽子,而是在描述,圣灵飞行、降落的姿态像鸽子一样。类似地,\textit{Stein}~\cite{book:stein1993luke}也不支持圣灵形象是鸽子(take form);类似前述的解释同样是\textit{Bock}~\cite{book:bock1994luke}所列举的几种解释中、不那么生硬的一种,但\textit{Bock}认为它仍然略显牵强。
  
  3. \cite{book:sproul2016reformation}提到雏鸽在旧约献祭系统里具有重要的地位。并且在献祭文化的传统中,有爱意(affectionate)与温和(gentle)的象征。但同样也要提到,这仅仅是一个参考,因为鸽子的含义在旧约里是时常变化的。
  
  4. 一个容易陷入谬误的见解是,从这段经文将鸽子和神联系起来。例如\cite{book:smith2009not, blog:Luke3BibltStudy, book:vision2017Catho}都主张鸽子是神的表记、或象征,并进而引用《圣经》,试图发现鸽子像神的美好、圣洁的特征来。然而神用造物将自己(的行为)启示给人,是为了让人能理解。但我们始终要铭记的是,无论神用什么样的造物来比喻、启示,只要神没有明确启示这造物分别为圣、是圣洁的,我们就不能擅自推断出这样的结论,更不要说将这造物和神本身联系起来。例如,\BibT{出}{19}\BibS{4}{‘我向埃及人所行的事,你们都看见了,且看见我如鹰将你们背在翅膀上,带来归我。}但是,神已经在律法启示,不洁净的动物有:\BibT{利}{11}\BibS{16}{鸵鸟、夜鹰、鱼鹰、鹰与其类;}。可见,前面的“如鹰”不过是一个单纯的比喻而已,超乎这个比喻之上什么意义也没有。我们对待神启示自身的经节也要持以谨慎和敬畏的态度,以免如同\BibT{利}{20}\BibS{23}{你们不可做甚么神像与我相配,不可为自己做金银的神像。},在心中用造物的形象来配神的形象。
  
  另外一个需要关注的点是神的话“我所喜悦的”。这句话,这个场景和将要发生的一切,都指着预言\BibT{赛}{42}\BibS{1}{看哪!我的仆人,我所扶持、所拣选、心里所喜悦的。我已将我的灵赐给他,他必将公理传给外邦。}宣告了耶稣是弥赛亚,这也是在众人的见证下发生的。}
  \litem{\BibS{23}{耶稣开头传道,年纪约有三十岁。依人看来,他是约瑟的儿子,约瑟是希里的儿子,}}
  \ritem{“依人看来”,有双重意思。它既表示约瑟并非耶稣的生父,也表示希里并非约瑟的生父。实际上,希里是马利亚的生父,约瑟因为和马利亚婚配,从而成为“希里的儿子”。由此结合\BibT[]{太}{1:1-17},确定了耶稣由(养)父辈、母辈追溯到“亚伯拉罕的后裔,大卫的子孙”和“女人的后裔”的正统性。}
  \litem{\BibT{路}{4}\BibS{1}{耶稣被圣灵充满,从约旦河回来,圣灵将他引到旷野,四十天受魔鬼的试探。}}
  \ritem{\Tafter{这里有福音问题\queref{cpt2:que:4}。}
    
  这里有两个值得讨论的问题。其一是圣灵充满,其二是试探的意义。
    
  在旧约就已经有不少被圣灵充满的例证。例如比撒列\BibT{出}{35}\BibS{31}{又以{\Lord}的灵充满了他,使他有智能、聪明、知识,能做各样的工;},但以理\BibT{但}{4}\BibS{8}{末后,那照我神的名,称为伯提沙撒的但以理来到我面前,他里头有圣神的灵……}。与新约不同的是,这里圣灵充满给予人智慧和能力,完成神交托的重要的工作,不会长久驻留在人身上。例如扫罗不守诫命,神厌弃了他,于是\BibT{撒上}{16}\BibS{14}{耶和华的灵离开扫罗,有恶魔从耶和华那里来扰乱他。}
    
  试探和试炼的区别,见于唐崇荣牧师\cite{blog:TrailTang}的解释。\BibT{雅}{1}\BibS{13}{人被试探,不可说:“我是被{\Lord}试探”,因为{\Lord}不能被恶试探,他也不试探人。}试探虽然出于人的罪和魔鬼,但是神能保护被试探的人,把恶意的试探转变成善意的试炼,把魔鬼毁坏人的目的转为造就人的试炼。
  
  全段讨论耶稣受试探,可以从两个角度入手来看待这一段经文。
  
  1. 与亚当和夏娃所受的试探类比,可以看出在此耶稣和人的始祖所面临的处境是完全同等的,彰显出耶稣的人性,但主耶稣的义坚不可摧。\Tafter{关于这一点,\textit{Mill}~\cite{book:mill2008five}指出,圣经启示的试探具有三个特性:肉体的情欲、眼目的情欲和今生的骄傲。有载于\BibT{约壹}{2}\BibS{16}{因为凡世界上的事,就像肉体的情欲,眼目的情欲,并今生的骄傲,都不是从父来的,乃是从世界来的。},而相对地,\BibT{创}{3}\BibS{6}{于是,女人见那棵树的果子好作食物,也悦人的眼目,且是可喜爱的,能使人有智能,就摘下果子来吃了;又给她丈夫,她丈夫也吃了。}“好作食物”是“肉体的情欲”,“悦人眼目”是眼目的情欲,“使人有智能”是今生的骄傲。}始祖没能抵挡住试探,暴露出自己悖逆神的恶性,但主耶稣自始至终都顺服神的教导,祂展示给我们一个真正的人的形象,更展示给我们什么是真正的义人,\Tafter{祂道成肉身,却不随从世界,满有爱父的心,正如经上的教导\BibT{约壹}{2}\BibS{15}{不要爱世界和世界上的事。人若爱世界,爱父的心就不在他里面了。}可以看出,下面列举的这些试探,今时的我们也很容易遇到,我们要思想主耶稣曾经的作为,从祂那里得到战胜试探的勇气和智慧。}
  
  2. 从律法的角度来看,耶稣受试探显示他顺服律法的智慧。以下魔鬼的每一个试探都在诱使主耶稣违背神的诫命,但主既成为人的样式,就完完全全顺服在神的律法下,在魔鬼的面前得胜。故而,他知道我们行出律法的难处,体察我们,我们靠自己不能行出义,但依靠主就能改变自己的生命,顺服神的教导。
  }
  \litem{\BibS{2}{那些日子没有吃甚么,日子满了,他就饿了。}}
  \ritem{\Tafter{这里有福音问题\queref{cpt2:que:5}。}}
  \litem{\BibS{3}{魔鬼对他说:“你若是{\Lord}的儿子,可以吩咐这块石头变成食物。”}}
  \ritem{\Tafter{魔鬼的试探有两次都以“你若是神的儿子”开头,设陷阱让主对“神的儿子”这一身份产生怀疑。有时我们甚至也会自问,或者被人诘问“你要是神的儿女,为什么……”看着主耶稣,我们就知道,争论、乃至怀疑自己是不是神的儿女,这本身就已经中了试探的圈套了。当全心全意,爱主我们的神。}}
  \litem{\BibS{4}{耶稣回答说:“经上记着说:‘人活着不是单靠食物,乃是靠{\Lord}口里所出的一切话。’”}}
  \ritem{\Tafter{这里有福音问题\queref{cpt2:que:6}。}
    
  律法有载:\BibT{申}{8}\BibS{3}{他苦炼你,任你饥饿,将你和你列祖所不认识的吗哪赐给你吃,使你知道人活着不是单靠食物,乃是靠耶和华口里所出的一切话。}
  
  \Tafter{这里显示主耶稣的顺服,因为祂可以行神迹得到食物,却宁愿把主权交到父神的手中。}
  
  神曾经领以色列民出埃及,但是全会众一遇到饥饿、口渴的境地就忍不住向神抱怨,第一次(\BibT[]{出}{15:24}),神把苦水变甜;第二次(\BibT[]{出}{16:2-3}),神赐下吗哪;第三次(\BibT[]{出}{17:3}),神让磐石里流出水来。此后神赐下法柜和法版,会众都承认了神的典章,但这抱怨还没有停止。第四次(\BibT[]{民}{11:4-6}),百姓嫌弃神的食物寡淡,神赐下食物,但也为他们贪欲的心施下了严厉的惩罚;第五次(\BibT[]{民}{20:3-5}),神替会众解渴,却也褫夺了摩西、亚伦这一代人进入应许之地的机会。这中间因为其他原因,会众试探神的次数不胜枚举。

  因此,看起来这里好像只是在怂恿主耶稣运用神的权能变出食物,以证明自己的神性,但如果主耶稣从了魔鬼的计谋,就无异于否定神的公义,以及在过往对以色列民为了食物抱怨的判决。主在这里的回答是精准、合宜的。以色列民为食物抱怨了五次,但主一次都没有过。}
  \litem{\BibS{5}{魔鬼又领他上了高山,霎时间把天下的万国都指给他看,}}
  \ritem{}
  \litem{\BibS{6}{对他说:“这一切权柄、荣华,我都要给你,因为这原是交付我的,我愿意给谁就给谁。}}
  \ritem{这里魔鬼指认天下的权柄在自己手上,是真话还是谎话?
  
  魔鬼虽然爱说谎,但从某种程度上,牠这里讲得并没有错。因为从亚当夏娃从牠的计谋犯罪开始,全天下就都被罪恶捆绑了,\BibT{罗}{6}\BibS{16}{岂不晓得你们献上自己作奴仆,顺从谁,就作谁的奴仆吗?或作罪的奴仆,以至于死;或作顺命的奴仆,以至成义。}\BibS{17}{感谢{\Lord}!因为你们从前虽然作罪的奴仆,现今却从心里顺服了所传给你们道理的模范。}\BibT{弗}{2}\BibS{2}{那时,你们在其中行事为人,随从今世的风俗,顺服空中掌权者的首领,就是现今在悖逆之子心中运行的邪灵。}\BibT{约壹}{5}\BibS{19}{我们知道我们是属{\Lord}的,全世界都卧在那恶者手下。}但这种权柄是属于阴间的,并不长久,因为归根结底权柄是神的,\BibT{诗}{22}\BibS{18}{因为国权是耶和华的;他是管理万国的。}魔鬼自以为大,为辖制住人得意,但到审判的时候,\BibT{启}{20}\BibS{10}{那迷惑他们的魔鬼被扔在硫磺的火湖里,就是兽和假先知所在的地方。他们必昼夜受痛苦,直到永永远远。}\BibS{14}{死亡和阴间也被扔在火湖里,这火湖就是第二次的死。}
  }
  \litem{\BibS{7}{你若在我面前下拜,这都要归你。”}}
  \ritem{\Tafter{这样的试探,是典型的敌基督的作为。类比到我们身上,就是“你要丢了跟从主的信心,跟从……,就会……”。这样的试探可能是随着利益的诱惑给我们的,也有可能是在折磨中给我们的。为的得到什么,就去违背诫命,甚至背主背道。这里已经明确启示,这是魔鬼的道路了。真的有一天,若是这试探临到,我们可能会非常痛苦,就连彼得也为了保全自己的性命,一时间不敢认主三次(\BibT[]{太}{26:69-75})。所以我们真的要祷告,求神将勇气加添给我们,能逃脱、或是抵挡住魔鬼的权势。}}
  \litem{\BibS{8}{耶稣说:“经上记着说:当拜主你的{\Lord},单要事奉他。”}}
  \ritem{律法有载:\BibT{申}{6}\BibS{13}{你要敬畏耶和华你的{\Lord},事奉他,指着他的名起誓。}
  
  这就是尊重神的主权,它合乎十诫的第一条(\BibT[]{出}{20:2-3})。主耶稣也明确给出教导,最大的诫命是\BibT{太}{22}\BibS{37}{耶稣对他说:“你要尽心、尽性、尽意,爱主你的{\Lord}。}不为权势所诱惑,也不屈服与权势,单单顺服神和祂的旨意,主耶稣为我们作出了表率。}
  \litem{\BibS{9}{魔鬼又领他到耶路撒冷去,叫他站在殿顶(注:“顶”原文作“翅”)上,对他说:“你若是{\Lord}的儿子,可以从这里跳下去;}}
  \ritem{\Tafter{这里有福音问题\queref{cpt2:que:7}}。}
  \litem{\BibS{10}{因为经上记着说:{}‘主要为你吩咐他的使者保护你。}}
  \ritem{\Tafter{这里显示魔鬼的策略变化。牠一开始尚且提出合理怀疑(饥饿),不成功马上转为说半真半假的谎(掌权),再不成就假冒神的教导。可见一切抵挡魔鬼的能力,终究归于理解《圣经》,只有明白经上的智慧,我们才能保护自己,不被魔鬼带到罪人的道路上。}}
  \litem{\BibS{11}{他们要用手托着你,免得你的脚碰在石头上。’”}}
  \ritem{这里显示魔鬼熟悉《圣经》,却同时狡猾地删改了原文,\BibT{诗}{91}\BibS{11}{因他要为你吩咐他的使者,在你行的一切道路上保护你。}\BibS{12}{他们要用手托着你,免得你的脚碰在石头上。}牠故意去掉了“在你行的一切道路上”,显然这和主站在殿顶的情况是不符的,魔鬼却用损害《圣经》原意的形式,来试探主。}
  \litem{\BibS{12}{耶稣对他说:“经上说:‘不可试探主你的{\Lord}。’”}}
  \ritem{但同时须知的是,即使不考虑与原意的出入,主的看护是时时同在的,也不能从魔鬼的计谋,因为主已经给出答案。律法有载:\BibT{申}{6}\BibS{16}{你们不可试探耶和华你们的{\Lord},……}
  
  相对地,以色列民却\BibT{民}{14}\BibS{22}{这些人虽看见我的荣耀和我在埃及与旷野所行的神迹,仍然试探我这十次,不听从我的话,}在新约里,我们还能陆续见到他们不断试探主,这从不信和罪中来的试探,实在是过于致命。实际上,任何时候,我们思想“神现在没有与我同在”或者“神掩面没有看我”,都是在试探神。而这种自我欺骗的后果,往往就是心里不信、行为上也干犯神。常常思想神的无所不在,常常怀有对神的敬畏,不站罪人的道路,主耶稣已经为我们作出表率。}
  \litem{\BibS{13}{魔鬼用完了各样的试探,就暂时离开耶稣。}}
  \ritem{我们重新思想这段经节:\BibT{来}{4}\BibS{15}{……他也曾凡事受过试探,与我们一样;只是他没有犯罪。}
    
  \cite{book:sproul2016reformation}这样解释圣灵领主受试探的意义:\emph{圣灵引导主耶稣进入旷野四十天,是要主耶稣经受这测试(test),它是指着亚当和以色列民的。这四十天的忍饥挨饿,指着以色列民在旷野里的四十年。主耶稣受到的三次试探,指着以色列民触犯的三条诫命。过去神的诫命以色列民没有守住,主耶稣守住了,祂用完全的顺服显示了祂确实是一个真正的“以色列人”。}
  
  试探临到时,主用律法的教导战胜了魔鬼,这是我们这些被罪辖制的人所做不到的。但它显示出律法真的是公义、正直、可喜爱的,令我们不由赞美主有福:\BibT{诗}{1}\BibS{2}{惟喜爱耶和华的律法,昼夜思想,这人便为有福!}
  
  主抵挡住了试探,不但没有犯罪,还教我们祷告,\BibT{路}{11}\BibS{4}{……不叫我们遇见试探;救我们脱离凶恶(注:有古卷无末句)}。魔鬼狡猾、凶恶,我们在牠面前真的软弱无力。可是,主不但战胜了牠,还愿做我们的盾牌,保我们“不受试探”。\Tafter{从而让我们明白,靠自己不能得救,宜当承认自己不配如此大的救恩,谦卑感恩,不靠自我修炼单靠神,从而在}当试探真的临到的时候,藉着对主的信心来战胜牠,甚至像约伯一样,将试探变为造就的试炼,将属世的咒诅变为属灵的祝福。

  特别值得注意的是,我们要怎样依靠神来脱离这试探。遭受痛苦、不幸时,我们很容易以自我为中心,陷入“我是受害者”的心态里无法自拔。这时,远离试探需要我们藉着对主的信心,离开这种低谷。让圣灵抚平我们的伤口,赐给我们力量饶恕冒犯我们的人,从自大、苦毒和长久的伤害里走出来。}
\end{paracol}

\qquad

\section{福音问题}

\begin{enumerate}
  \item \label{cpt2:que:1}这位耶稣凡事受过试探,却没有犯罪,但是圣经上却记载祂被钉死在十字架上,祂为了什么?
  \begin{itemize}
    \item \textbf{遵从主命,替人受过}:\BibT[]{赛}{53:4-12},其中明言\BibS{9}{他虽然未行强暴,口中也没有诡诈,人还使他与恶人同埋;谁知死的时候与财主同葬。}\BibS{10}{耶和华却定意(注:或作“喜悦”)将他压伤,使他受痛苦;耶和华以他为赎罪祭(注:或作“他献本身为赎罪祭”)。他必看见后裔,并且延长年日,耶和华所喜悦的事必在他手中亨通。}又有\BibT{腓}{2}\BibS{8}{既有人的样子,就自己卑微,存心顺服,以至于死,且死在十字架上。}
    \item \textbf{战胜罪}:\BibT{罗}{6}\BibS{6}{因为知道我们的旧人和他同钉十字架,使罪身灭绝,叫我们不再作罪的奴仆,}
    \item \textbf{人的愚昧}: \BibT{林前}{2}\BibS{8}{这智能,世上有权有位的人没有一个知道的;他们若知道,就不把荣耀的主钉在十字架上了。}
    \item \textbf{救人与神和好}: \BibT{西}{1}\BibS{20}{既然借着他在十字架上所流的血成就了和平,便借着他叫万有,无论是地上的、天上的,都与自己和好了。}
    \item \textbf{医治世人}: \BibT{彼前}{2}\BibS{24}{他被挂在木头上,亲身担当了我们的罪,使我们既然在罪上死,就得以在义上活。因他受的鞭伤,你们便得了医治。}
  \end{itemize}
  \item \label{cpt2:que:2}这位耶稣知道你一切的软弱,一切的痛苦,你是否愿意悔改,归向祂?
  \begin{itemize}
    \item \BibT{来}{4}\BibS{15}{因我们的大祭司并非不能体恤我们的软弱。他也曾凡事受过试探,与我们一样,只是他没有犯罪。}\BibS{16}{所以,我们只管坦然无惧的来到施恩的宝座前,为要得怜恤,蒙恩惠,作随时的帮助。}
  \end{itemize}
  \item \label{cpt2:que:3}这里主被圣灵充满有什么涵义?或者说有什么意义?
  \begin{itemize}
    \item 须知主耶稣本就是道,是神,祂与圣灵本就同在。这里看似是一件“没有必要发生的事”,但其实不然。这个场景完全印证了\BibT[]{赛}{42:1}(正文解释里有引)的预言,是要当时在场的人、这经本身,以及我们这些人,为主耶稣是弥赛亚、是神作见证。\BibT{约}{5}\BibS{39}{你们查考圣经(注:或作“应当查考圣经”),因你们以为内中有永生;给我作见证的就是这经。}
  \end{itemize}
  \item \label{cpt2:que:4}魔鬼是谁?是撒旦吗?
  \begin{itemize}
    \item 是的,有载\BibT{启}{20}\BibS{2}{他捉住那龙,就是古蛇,又叫魔鬼,也叫撒但,把牠捆绑一千年,}
  \end{itemize}
  \item \label{cpt2:que:5}主耶稣受试探是发生在什么时候?
  \begin{itemize}
    \item 有人可能会因为\BibT[]{v}{2}的这段经文感到疑惑,难道主耶稣是在四十天满的时候才开始受到魔鬼的试探吗?从对照经节来看,\BibT{可}{1}\BibS{13}{他在旷野四十天,受撒但的试探,并与野兽同在一处;且有天使来伺候他。}主耶稣受试探的经历应在这四十天以内。
  \end{itemize}
  \item \label{cpt2:que:6}“靠着神的话活着”对我们的生活有什么意义?
  \begin{itemize}
    \item 主耶稣要用神迹使自己饱足是轻而易举的,但祂没有这样做。这是缘于祂从不行神迹来满足自己的私欲。无可否认,人在饥饿的时候确实是容易失节的,但仍要守神的道,看见神的掌权,不靠自己而依靠神。\BibT{林后}{1}\BibS{9}{自己心里也断定是必死的,叫我们不靠自己,只靠叫死人复活的{\Lord}。}只为食物活着是不合宜的,宜为神的道活着,因为我们记念神的作为:\BibT{诗}{66}\BibS{9}{他使我们的性命存活,也不叫我们的脚摇动。}有了神的话,常常思想,我们就不致于因物质失节,从而常常自问,“比起物质生活,我是不是更在意神的话语?”
  \end{itemize}
  \item \label{cpt2:que:7}魔鬼为什么带主耶稣到殿顶,才指使他跳下去?
  \begin{itemize}
    \item \textit{Bock}~\cite{book:bock1994luke}的解释是,这里的殿虽然不是特指圣殿,但在耶路撒冷内,是离神的住所近的地方,以便加强“神吩咐使者保护你”的预设,仿佛在说“跳下去是不会有危险的,还能印证神的话”。而显然,这种“求印证”的做法是不合宜的,主耶稣已经明言“不可试探神”。
  \end{itemize}
\end{enumerate}

\renewcommand{\bibname}{本章参考}
\bibliographystyle{IEEEtran}
\bibliography{bib/tex201801}